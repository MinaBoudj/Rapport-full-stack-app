\chapter*{Glossaire}
\addcontentsline{toc}{chapter}{Glossaire}

\begin{minipage}{\textwidth}
    \begin{minipage}{0.1\textwidth}
        \textbf{IDE}
    \end{minipage}\hfill
    \begin{minipage}{0.8\textwidth}
        Environnement de développement, un logiciel qui fournit des outils pour développer des logiciels.
    \end{minipage}
\end{minipage}

\vspace{1cm}

\begin{minipage}{\textwidth}
    \begin{minipage}{0.1\textwidth}
        \textbf{Git}
    \end{minipage}\hfill
    \begin{minipage}{0.8\textwidth}
        Logiciel de gestion de versions, qui permet de suivre les modifications de code, de gérer les branches et de collaborer avec d'autres développeurs.
    \end{minipage}
\end{minipage}

\vspace{1cm}

\begin{minipage}{\textwidth}
    \begin{minipage}{0.1\textwidth}
        \textbf{Stack}
    \end{minipage}\hfill
    \begin{minipage}{0.8\textwidth}
        Ensemble de technologies et outils utilisés pour construire une application.
    \end{minipage}
\end{minipage}

\vspace{1cm}

\begin{minipage}{\textwidth}
    \begin{minipage}{0.1\textwidth}
        \textbf{UML}
    \end{minipage}\hfill
    \begin{minipage}{0.8\textwidth}
        Unified Modeling Language, Langage de modélisation graphique et textuel pour visualiser les artefacts d’un système logiciel.
    \end{minipage}
\end{minipage}

\vspace{1cm}

\begin{minipage}{\textwidth}
    \begin{minipage}{0.1\textwidth}
        \textbf{Backlog}
    \end{minipage}\hfill
    \begin{minipage}{0.8\textwidth}
        Liste ordonnée de tâches ou de fonctionnalités à réaliser dans un projet.
    \end{minipage}
\end{minipage}

\vspace{1cm}

\begin{minipage}{\textwidth}
    \begin{minipage}{0.1\textwidth}
        \textbf{Sprint}
    \end{minipage}\hfill
    \begin{minipage}{0.8\textwidth}
        Itération de développement dans Agile, durant laquelle une équipe travaille sur des tâches spécifiques du backlog.
    \end{minipage}
\end{minipage}

\vspace{1cm}

\begin{minipage}{\textwidth}
    \begin{minipage}{0.1\textwidth}
        \textbf{API}
    \end{minipage}\hfill
    \begin{minipage}{0.8\textwidth}
        Application Programming Interface, Interface qui sert de façade pour les logiciels qui l'utilisent.
    \end{minipage}
\end{minipage}

\vspace{1cm}

\begin{minipage}{\textwidth}
    \begin{minipage}{0.1\textwidth}
        \textbf{Docker}
    \end{minipage}\hfill
    \begin{minipage}{0.8\textwidth}
        Plateforme open source automatiser le déploiement et la gestion d’applications dans des conteneurs, rendant les applications portables et cohérentes à travers différents environnements.
    \end{minipage}
\end{minipage}

\vspace{1cm}

\begin{minipage}{\textwidth}
    \begin{minipage}{0.1\textwidth}
        \textbf{NGINX}
    \end{minipage}\hfill
    \begin{minipage}{0.8\textwidth}
        Serveur web open source, léger et rapide, utilisé pour servir des sites web et gérer les connexions HTTP.
    \end{minipage}
\end{minipage}

\vspace{1cm}

\begin{minipage}{\textwidth}
    \begin{minipage}{0.1\textwidth}
        \textbf{Kubernetes}
    \end{minipage}\hfill
    \begin{minipage}{0.8\textwidth}
        Plateforme open source pour la gestion d'applications dans des conteneurs, permettant de déployer des services.
    \end{minipage}
\end{minipage}

\vspace{1cm}

\begin{minipage}{\textwidth}
    \begin{minipage}{0.1\textwidth}
        \textbf{Mapper}
    \end{minipage}\hfill
    \begin{minipage}{0.8\textwidth}
        Mettre en correspondance les champs de plusieurs bases dedonnées
    \end{minipage}
\end{minipage}

\vspace{1cm}

\begin{minipage}{\textwidth}
    \begin{minipage}{0.1\textwidth}
        \textbf{URL}
    \end{minipage}\hfill
    \begin{minipage}{0.8\textwidth}
        Uniform Resource Locator, identifie de manière unique une ressource sur Internet.
    \end{minipage}
\end{minipage}

\vspace{1cm}

\begin{minipage}{\textwidth}
    \begin{minipage}{0.1\textwidth}
        \textbf{Framework}
    \end{minipage}\hfill
    \begin{minipage}{0.8\textwidth}
        Ensemble de composants logiciels structurels pour développer des logiciels. 
    \end{minipage}
\end{minipage}

\vspace{1cm}

\begin{minipage}{\textwidth}
    \begin{minipage}{0.1\textwidth}
        \textbf{JSON}
    \end{minipage}\hfill
    \begin{minipage}{0.8\textwidth}
        Java Script Object Notation, format de texte utilisé pour représenter les données structurées basées sur la syntaxe des objets JavaScript.
    \end{minipage}
\end{minipage}

\vspace{1cm}

\begin{minipage}{\textwidth}
    \begin{minipage}{0.1\textwidth}
        \textbf{JPA}
    \end{minipage}\hfill
    \begin{minipage}{0.8\textwidth}
        Java Persistence API, spécification Java pour gérer les données relationnelles via le mapping objet-relationnel (ORM). Facilite l’interaction entre les objets Java et les bases de données.
    \end{minipage}
\end{minipage}

\vspace{1cm}

\begin{minipage}{\textwidth}
    \begin{minipage}{0.1\textwidth}
        \textbf{CRUD}
    \end{minipage}\hfill
    \begin{minipage}{0.8\textwidth}
        Create Read Upadate Delete, acronyme qui désigne les qu’âtres opérations de base utilisées dans la gestion des données persistantes.    
    \end{minipage}
\end{minipage}

\vspace{1cm}

\begin{minipage}{\textwidth}
    \begin{minipage}{0.1\textwidth}
        \textbf{ORM}
    \end{minipage}\hfill
    \begin{minipage}{0.8\textwidth}
        Object Relational Mapping, technique permettant de convertir les données entre les systèmes de type objet et les bases de données relationnelles. Hibernate est un exemple populaire d’ORM.    
    \end{minipage}
\end{minipage}

\vspace{1cm}

\begin{minipage}{\textwidth}
    \begin{minipage}{0.1\textwidth}
        \textbf{DTO}
    \end{minipage}\hfill
    \begin{minipage}{0.8\textwidth}
        Data Transfer Object, Objet utilisé pour transférer des données entre différentes couches d’une application.
    \end{minipage}
\end{minipage}

\vspace{1cm}

\begin{minipage}{\textwidth}
    \begin{minipage}{0.1\textwidth}
        \textbf{DAO}
    \end{minipage}\hfill
    \begin{minipage}{0.8\textwidth}
        Data Access Object, Modèle de conception fournissant une abstraction pour les opérations CRUD sur les entités.    
    \end{minipage}
\end{minipage}

\vspace{1cm}

\begin{minipage}{\textwidth}
    \begin{minipage}{0.1\textwidth}
        \textbf{HTTP}
    \end{minipage}\hfill
    \begin{minipage}{0.8\textwidth}
        Hypertext Transfer Protocol, est un protocole de transfert de données dans le web. 
    \end{minipage}
\end{minipage}

\vspace{1cm}

\begin{minipage}{\textwidth}
    \begin{minipage}{0.1\textwidth}
        \textbf{GET}
    \end{minipage}\hfill
    \begin{minipage}{0.8\textwidth}
        Méthode HTTP qui demande des données à partir d'un serveur.
    \end{minipage}
\end{minipage}

\vspace{1cm}

\begin{minipage}{\textwidth}
    \begin{minipage}{0.1\textwidth}
        \textbf{POST}
    \end{minipage}\hfill
    \begin{minipage}{0.8\textwidth}
        Méthode HTTP qui envoie les données au serveur.
    \end{minipage}
\end{minipage}
