\chapter*{Contexte du projet}
\addcontentsline{toc}{chapter}{Contexte du projet}

L’objectif du projet est de créer une application web qui permet aux agents de l’entreprise de noter les formations qu’ils ont suivi.
\medskip

La CNIEG souhaite encourager les agents à être acteurs de leur parcours professionnel et à identifier les formations qui répondent le mieux à leurs besoins. L'objectif est de créer un outil collaboratif qui permet aux agents de partager leurs expériences et leurs avis sur les formations suivies. 
Nous avons pris contact avec le service Ressources Humaines pour définir les besoins fonctionnels et les caractéristiques de l’application.
\medskip

Spécification des besoins fonctionnels :
\begin{enumerate}[leftmargin=2cm]
    \item \textbf{Donner son avis :} les agents peuvent donner leurs avis sur les formations qu'ils ont suivies, en évaluant différents aspects de la formation (organisation, durée, contenu, etc.).
    \item \textbf{Consulter les avis :} les agents peuvent consulter les avis des autres collègues sur les formations suivies, pour s'inspirer et vérifier la qualité de la formation dispensée.
\end{enumerate}
\medskip

Caractéristiques :
\medskip
\begin{itemize}[leftmargin=2cm]
    \item L'application est accessible à tous les agents de la CNIEG.
    \item L'application est complémentaire de l'évaluation à froid adressée par le pôle RH 3 à 4 mois après les formations.
    \item Le lien de l'application est envoyé par mail après chaque formation réalisée, et est également disponible sur le SharePoint - espace salarié.
\end{itemize}
\medskip

Items d'évaluation :
\medskip
\begin{itemize}[leftmargin=2cm]
    \item Organisation de la formation (convocation, plan d'accès, lien de connexion si formation à distace, locaux...)
    \item Durée de la formation.
    \item Avez-vous appris des choses nouvelles ?
    \item La formation a-t-elle répondu à vos attentes ?
    \item Etc. (items à définir)
\end{itemize}
\medskip

Système d'évaluation :
\medskip
\begin{itemize}[leftmargin=2cm]
    \item Chaque item d'évaluation a un nombre d'étoiles associé (entre 1 et 5).
    \item Chaque item d'évaluation a une zone de commentaire libre.
\end{itemize}